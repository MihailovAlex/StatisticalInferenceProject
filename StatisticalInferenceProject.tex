% Options for packages loaded elsewhere
\PassOptionsToPackage{unicode}{hyperref}
\PassOptionsToPackage{hyphens}{url}
%
\documentclass[
]{article}
\usepackage{amsmath,amssymb}
\usepackage{iftex}
\ifPDFTeX
  \usepackage[T1]{fontenc}
  \usepackage[utf8]{inputenc}
  \usepackage{textcomp} % provide euro and other symbols
\else % if luatex or xetex
  \usepackage{unicode-math} % this also loads fontspec
  \defaultfontfeatures{Scale=MatchLowercase}
  \defaultfontfeatures[\rmfamily]{Ligatures=TeX,Scale=1}
\fi
\usepackage{lmodern}
\ifPDFTeX\else
  % xetex/luatex font selection
\fi
% Use upquote if available, for straight quotes in verbatim environments
\IfFileExists{upquote.sty}{\usepackage{upquote}}{}
\IfFileExists{microtype.sty}{% use microtype if available
  \usepackage[]{microtype}
  \UseMicrotypeSet[protrusion]{basicmath} % disable protrusion for tt fonts
}{}
\makeatletter
\@ifundefined{KOMAClassName}{% if non-KOMA class
  \IfFileExists{parskip.sty}{%
    \usepackage{parskip}
  }{% else
    \setlength{\parindent}{0pt}
    \setlength{\parskip}{6pt plus 2pt minus 1pt}}
}{% if KOMA class
  \KOMAoptions{parskip=half}}
\makeatother
\usepackage{xcolor}
\usepackage[margin=1in]{geometry}
\usepackage{color}
\usepackage{fancyvrb}
\newcommand{\VerbBar}{|}
\newcommand{\VERB}{\Verb[commandchars=\\\{\}]}
\DefineVerbatimEnvironment{Highlighting}{Verbatim}{commandchars=\\\{\}}
% Add ',fontsize=\small' for more characters per line
\usepackage{framed}
\definecolor{shadecolor}{RGB}{248,248,248}
\newenvironment{Shaded}{\begin{snugshade}}{\end{snugshade}}
\newcommand{\AlertTok}[1]{\textcolor[rgb]{0.94,0.16,0.16}{#1}}
\newcommand{\AnnotationTok}[1]{\textcolor[rgb]{0.56,0.35,0.01}{\textbf{\textit{#1}}}}
\newcommand{\AttributeTok}[1]{\textcolor[rgb]{0.13,0.29,0.53}{#1}}
\newcommand{\BaseNTok}[1]{\textcolor[rgb]{0.00,0.00,0.81}{#1}}
\newcommand{\BuiltInTok}[1]{#1}
\newcommand{\CharTok}[1]{\textcolor[rgb]{0.31,0.60,0.02}{#1}}
\newcommand{\CommentTok}[1]{\textcolor[rgb]{0.56,0.35,0.01}{\textit{#1}}}
\newcommand{\CommentVarTok}[1]{\textcolor[rgb]{0.56,0.35,0.01}{\textbf{\textit{#1}}}}
\newcommand{\ConstantTok}[1]{\textcolor[rgb]{0.56,0.35,0.01}{#1}}
\newcommand{\ControlFlowTok}[1]{\textcolor[rgb]{0.13,0.29,0.53}{\textbf{#1}}}
\newcommand{\DataTypeTok}[1]{\textcolor[rgb]{0.13,0.29,0.53}{#1}}
\newcommand{\DecValTok}[1]{\textcolor[rgb]{0.00,0.00,0.81}{#1}}
\newcommand{\DocumentationTok}[1]{\textcolor[rgb]{0.56,0.35,0.01}{\textbf{\textit{#1}}}}
\newcommand{\ErrorTok}[1]{\textcolor[rgb]{0.64,0.00,0.00}{\textbf{#1}}}
\newcommand{\ExtensionTok}[1]{#1}
\newcommand{\FloatTok}[1]{\textcolor[rgb]{0.00,0.00,0.81}{#1}}
\newcommand{\FunctionTok}[1]{\textcolor[rgb]{0.13,0.29,0.53}{\textbf{#1}}}
\newcommand{\ImportTok}[1]{#1}
\newcommand{\InformationTok}[1]{\textcolor[rgb]{0.56,0.35,0.01}{\textbf{\textit{#1}}}}
\newcommand{\KeywordTok}[1]{\textcolor[rgb]{0.13,0.29,0.53}{\textbf{#1}}}
\newcommand{\NormalTok}[1]{#1}
\newcommand{\OperatorTok}[1]{\textcolor[rgb]{0.81,0.36,0.00}{\textbf{#1}}}
\newcommand{\OtherTok}[1]{\textcolor[rgb]{0.56,0.35,0.01}{#1}}
\newcommand{\PreprocessorTok}[1]{\textcolor[rgb]{0.56,0.35,0.01}{\textit{#1}}}
\newcommand{\RegionMarkerTok}[1]{#1}
\newcommand{\SpecialCharTok}[1]{\textcolor[rgb]{0.81,0.36,0.00}{\textbf{#1}}}
\newcommand{\SpecialStringTok}[1]{\textcolor[rgb]{0.31,0.60,0.02}{#1}}
\newcommand{\StringTok}[1]{\textcolor[rgb]{0.31,0.60,0.02}{#1}}
\newcommand{\VariableTok}[1]{\textcolor[rgb]{0.00,0.00,0.00}{#1}}
\newcommand{\VerbatimStringTok}[1]{\textcolor[rgb]{0.31,0.60,0.02}{#1}}
\newcommand{\WarningTok}[1]{\textcolor[rgb]{0.56,0.35,0.01}{\textbf{\textit{#1}}}}
\usepackage{graphicx}
\makeatletter
\def\maxwidth{\ifdim\Gin@nat@width>\linewidth\linewidth\else\Gin@nat@width\fi}
\def\maxheight{\ifdim\Gin@nat@height>\textheight\textheight\else\Gin@nat@height\fi}
\makeatother
% Scale images if necessary, so that they will not overflow the page
% margins by default, and it is still possible to overwrite the defaults
% using explicit options in \includegraphics[width, height, ...]{}
\setkeys{Gin}{width=\maxwidth,height=\maxheight,keepaspectratio}
% Set default figure placement to htbp
\makeatletter
\def\fps@figure{htbp}
\makeatother
\setlength{\emergencystretch}{3em} % prevent overfull lines
\providecommand{\tightlist}{%
  \setlength{\itemsep}{0pt}\setlength{\parskip}{0pt}}
\setcounter{secnumdepth}{-\maxdimen} % remove section numbering
\ifLuaTeX
  \usepackage{selnolig}  % disable illegal ligatures
\fi
\usepackage{bookmark}
\IfFileExists{xurl.sty}{\usepackage{xurl}}{} % add URL line breaks if available
\urlstyle{same}
\hypersetup{
  pdftitle={StatisticalInferenceProject},
  pdfauthor={Alex Mihailov},
  hidelinks,
  pdfcreator={LaTeX via pandoc}}

\title{StatisticalInferenceProject}
\author{Alex Mihailov}
\date{2024-09-04}

\begin{document}
\maketitle

\subsection{Steps}\label{steps}

\begin{enumerate}
\def\labelenumi{\arabic{enumi}.}
\tightlist
\item
  Show the sample mean and compare it to the theoretical mean of the
  distribution.
\item
  Show how variable the sample is (via variance) and compare it to the
  theoretical variance of the distribution.
\item
  Show that the distribution is approximately normal.
\end{enumerate}

\subsection{Data}\label{data}

\begin{Shaded}
\begin{Highlighting}[]
\FunctionTok{library}\NormalTok{(}\StringTok{"data.table"}\NormalTok{)}
\FunctionTok{library}\NormalTok{(}\StringTok{"ggplot2"}\NormalTok{)}

\FunctionTok{set.seed}\NormalTok{(}\DecValTok{100}\NormalTok{)}

\NormalTok{lambda }\OtherTok{\textless{}{-}} \FloatTok{0.2}

\NormalTok{n }\OtherTok{\textless{}{-}} \DecValTok{30}

\NormalTok{simulations }\OtherTok{\textless{}{-}} \DecValTok{500}

\NormalTok{simulated\_exponentials }\OtherTok{\textless{}{-}} \FunctionTok{replicate}\NormalTok{(simulations, }\FunctionTok{rexp}\NormalTok{(n, lambda))}

\NormalTok{means\_exponentials }\OtherTok{\textless{}{-}} \FunctionTok{apply}\NormalTok{(simulated\_exponentials, }\DecValTok{2}\NormalTok{, mean)}
\end{Highlighting}
\end{Shaded}

\subsection{Question 1}\label{question-1}

Show where the distribution is centered at and compare it to the
theoretical center of the distribution

\begin{Shaded}
\begin{Highlighting}[]
\NormalTok{analytical\_mean }\OtherTok{\textless{}{-}} \FunctionTok{mean}\NormalTok{(means\_exponentials)}
\NormalTok{analytical\_mean}
\end{Highlighting}
\end{Shaded}

\begin{verbatim}
## [1] 5.041372
\end{verbatim}

Mean

\begin{Shaded}
\begin{Highlighting}[]
\NormalTok{theory\_mean }\OtherTok{\textless{}{-}} \DecValTok{1}\SpecialCharTok{/}\NormalTok{lambda}
\NormalTok{theory\_mean}
\end{Highlighting}
\end{Shaded}

\begin{verbatim}
## [1] 5
\end{verbatim}

visualization

\begin{Shaded}
\begin{Highlighting}[]
\FunctionTok{hist}\NormalTok{(means\_exponentials, }\AttributeTok{xlab =} \StringTok{"mean"}\NormalTok{, }\AttributeTok{main =} \StringTok{"Exponential function simulations"}\NormalTok{)}
\FunctionTok{abline}\NormalTok{(}\AttributeTok{v =}\NormalTok{ analytical\_mean, }\AttributeTok{col =} \StringTok{"red"}\NormalTok{)}
\FunctionTok{abline}\NormalTok{(}\AttributeTok{v =}\NormalTok{ theory\_mean, }\AttributeTok{col =} \StringTok{"blue"}\NormalTok{)}
\end{Highlighting}
\end{Shaded}

\includegraphics{StatisticalInferenceProject_files/figure-latex/unnamed-chunk-4-1.pdf}

The analytics mean is 4.97335 the theoretical mean 5. The center of
distribution of averages of 30 exponentials is very close to the
theoretical center of the distribution.

\subsection{Question 2}\label{question-2}

Show how variable it is and compare it to the theoretical variance of
the distribution

Standard deviation of distribution

\begin{Shaded}
\begin{Highlighting}[]
\NormalTok{standard\_deviation\_dist }\OtherTok{\textless{}{-}} \FunctionTok{sd}\NormalTok{(means\_exponentials)}
\NormalTok{standard\_deviation\_dist}
\end{Highlighting}
\end{Shaded}

\begin{verbatim}
## [1] 0.9316163
\end{verbatim}

Standard deviation from analytical expression

\begin{Shaded}
\begin{Highlighting}[]
\NormalTok{standard\_deviation\_theory }\OtherTok{\textless{}{-}}\NormalTok{ (}\DecValTok{1}\SpecialCharTok{/}\NormalTok{lambda)}\SpecialCharTok{/}\FunctionTok{sqrt}\NormalTok{(n)}
\NormalTok{standard\_deviation\_theory}
\end{Highlighting}
\end{Shaded}

\begin{verbatim}
## [1] 0.9128709
\end{verbatim}

Variance of distribution

\begin{Shaded}
\begin{Highlighting}[]
\NormalTok{variance\_dist }\OtherTok{\textless{}{-}}\NormalTok{ standard\_deviation\_dist}\SpecialCharTok{\^{}}\DecValTok{2}
\NormalTok{variance\_dist}
\end{Highlighting}
\end{Shaded}

\begin{verbatim}
## [1] 0.867909
\end{verbatim}

Variance from analytical expression

\begin{Shaded}
\begin{Highlighting}[]
\NormalTok{variance\_theory }\OtherTok{\textless{}{-}}\NormalTok{ ((}\DecValTok{1}\SpecialCharTok{/}\NormalTok{lambda)}\SpecialCharTok{*}\NormalTok{(}\DecValTok{1}\SpecialCharTok{/}\FunctionTok{sqrt}\NormalTok{(n)))}\SpecialCharTok{\^{}}\DecValTok{2}
\NormalTok{variance\_theory}
\end{Highlighting}
\end{Shaded}

\begin{verbatim}
## [1] 0.8333333
\end{verbatim}

Standard Deviation of the distribution is 0.9580265 with the theoretical
SD calculated as 0.9128709. The Theoretical variance is 0.9178148. The
actual variance of the distribution is 0.8333333

\subsection{Question 3}\label{question-3}

Show that the distribution is approximately normal

\begin{Shaded}
\begin{Highlighting}[]
\NormalTok{xfit }\OtherTok{\textless{}{-}} \FunctionTok{seq}\NormalTok{(}\FunctionTok{min}\NormalTok{(means\_exponentials), }\FunctionTok{max}\NormalTok{(means\_exponentials), }\AttributeTok{length=}\DecValTok{50}\NormalTok{)}
\NormalTok{yfit }\OtherTok{\textless{}{-}} \FunctionTok{dnorm}\NormalTok{(xfit, }\AttributeTok{mean=}\DecValTok{1}\SpecialCharTok{/}\NormalTok{lambda, }\AttributeTok{sd=}\NormalTok{(}\DecValTok{1}\SpecialCharTok{/}\NormalTok{lambda}\SpecialCharTok{/}\FunctionTok{sqrt}\NormalTok{(n)))}
\FunctionTok{hist}\NormalTok{(means\_exponentials,}\AttributeTok{breaks=}\NormalTok{n,}\AttributeTok{prob=}\NormalTok{T,}\AttributeTok{col=}\StringTok{"yellow"}\NormalTok{,}\AttributeTok{xlab =} \StringTok{"means"}\NormalTok{,}\AttributeTok{main=}\StringTok{"Density of means"}\NormalTok{,}\AttributeTok{ylab=}\StringTok{"density"}\NormalTok{)}
\FunctionTok{lines}\NormalTok{(xfit, yfit, }\AttributeTok{pch=}\DecValTok{30}\NormalTok{, }\AttributeTok{col=}\StringTok{"brown"}\NormalTok{, }\AttributeTok{lty=}\DecValTok{5}\NormalTok{)}
\end{Highlighting}
\end{Shaded}

\includegraphics{StatisticalInferenceProject_files/figure-latex/unnamed-chunk-9-1.pdf}

\begin{Shaded}
\begin{Highlighting}[]
\FunctionTok{qqnorm}\NormalTok{(means\_exponentials)}
\FunctionTok{qqline}\NormalTok{(means\_exponentials, }\AttributeTok{col =} \DecValTok{2}\NormalTok{)}
\end{Highlighting}
\end{Shaded}

\includegraphics{StatisticalInferenceProject_files/figure-latex/unnamed-chunk-10-1.pdf}

Due to Due to the central limit theorem (CLT), the distribution of
averages of 40 exponentials is very close to a normal distribution.

\end{document}
